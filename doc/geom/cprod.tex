\begin{tcolorbox}
    \begin{enumerate}
        \item Let $\mathbf{u} = (a, b)$, $\mathbf{v} = (c, d)$, and $\theta$ be the angle from $\mathbf{u}$ to $\mathbf{v}$ measured in counterclockwise direction. $(0 \leq \theta < 2\pi)$
        \item The signed area of the parallelogram formed by $\mathbf{u}$ and $\mathbf{v}$ is equal to $ad-bc$, and is notated as $\mathbf{u} \times \mathbf{v}$.
    \end{enumerate}
\end{tcolorbox}

It suffices to show that $ad-bc = |\mathbf{u}||\mathbf{v}|\sin\theta$. First, we use the rotation matrix to find a different expression for $\mathbf{v}$.

\begin{equation*}
    \mathbf{v}
    = \sqrt{\frac{c^2+d^2}{a^2+b^2}}
    \begin{pmatrix}
        \cos\theta & -\sin\theta \\
        \sin\theta & \cos\theta
    \end{pmatrix}
    \begin{pmatrix}
        a \\
        b
    \end{pmatrix}
    = \sqrt{\frac{c^2+d^2}{a^2+b^2}}
    \begin{pmatrix}
        a\cos\theta - b\sin\theta \\
        a\sin\theta + b\cos\theta
    \end{pmatrix}
\end{equation*}

Then we calculate $ad-bc$ using the alternate expression.

\begin{align*}
    ad-bc = {}& \sqrt{\frac{c^2+d^2}{a^2+b^2}}(a(a\sin\theta + b\cos\theta) - b(a\cos\theta - b\sin\theta)) \\
    = {}& \sqrt{\frac{c^2+d^2}{a^2+b^2}}(a^2\sin\theta + b^2\sin\theta) \\
    = {}& \sqrt{a^2+b^2}\sqrt{c^2+d^2}\sin\theta \\
    = {}& |\mathbf{u}||\mathbf{v}|\sin\theta
\end{align*}

This fact implies that $\mathbf{v}$ is oriented counterclockwise to $\mathbf{u}$ if $\mathbf{u} \times \mathbf{v}$ is positive, and clockwise if $\mathbf{u} \times \mathbf{v}$ is negative. The cross product of zero indicates that the two vectors are parallel to each other.