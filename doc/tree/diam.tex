{\scriptsize If the weights can be negative, then this algorithm does not work in the general case.}

\begin{tcolorbox}
    \begin{enumerate}
        \item Let $T$ be a given tree with non-negative weights, and $p$ be an arbitrary node on $T$.
        \item Let $a$ be any farthest node from $p$.
        \item Let $b$ be any farthest node from $a$.
        \item The path from $a$ to $b$ is one of the diameters of $T$.
    \end{enumerate}
\end{tcolorbox}

Let $P$ denote the path from $a$ to $b$. For each vertex $v$, define $r_v$ as the first \emph{reachable} vertex on $P$ from $v$. First, we show that for every vertex $v$, $\dist(r_v, v) \leq \dist(r_v, b)$. By definition of $b$, $\dist(a, v) \leq \dist(a, b)$. Subtracting $\dist(a, r_v)$ on both sides immediately yields $\dist(r_v, v) \leq \dist(r_v, b)$.

\medskip

Next, we proceed to show $\dist(r_v, v) \leq \dist(r_v, a)$ for every vertex $v$. Given an arbitrary $v$, the position of $r_v$ splits into two cases.

\[
\begin{cases}
    \text{$r_v$ exists on the path from $a$ (\emph{inclusive}) to $r_p$ (\emph{exclusive}).} \\
    \text{$r_v$ exists on the path from $r_p$ (\emph{inclusive}) to $b$ (\emph{inclusive}).} \\
\end{cases}
\]

In the former case, $\dist(r_v, v) \leq \dist(r_v, a)$ must hold, because otherwise it contradicts the fact that $a$ is farthest from $p$. In the latter case, $\dist(r_v, b) \leq \dist(r_v, a)$, because otherwise $b$ is farther away from $p$ than $a$. Combining this with the previously derived inequality $\dist(r_v, v) \leq \dist(r_v, b)$ returns $\dist(r_v, v) \leq \dist(r_v, a)$.

\medskip

Finally, we show that $\dist(u, v) \leq \dist(a, b)$ for every pair of vertices $u$ and $v$. Without loss of generality, assume that $r_u$ exists on the path from $a$ to $r_v$.

\begin{align*}
    \dist(u, v) \leq {}& \dist(u, r_u) + \dist(r_u, r_v) + \dist(r_v, v) \\
    \leq {}& \dist(a, r_u) + \dist(r_u, r_v) + \dist(r_v, b) \\
    = {}& \dist(a, b)
\end{align*}

We have proven that $\dist(a, b)$ is greater than or equal to every $\dist(u, v)$; hence the algorithm correctly finds the diameter of $T$.